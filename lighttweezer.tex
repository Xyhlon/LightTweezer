%! TeX program = lualatex
%---------------------------ALLGEMEINE IMPORTS-------------------------------------
\documentclass[12pt,english,ngerman]{scrartcl}
\input{protokoll_template/template.latex/input/shared_preamble.tex}

% Kopfzeile
\ihead{WS22\\
	19.04.2023}

\chead{\textsc{Stark} Matthias --- 12004907 \\
	\textsc{Philipp} Maximilian --- 11839611}

\ohead{FLAB 2 \\
	Advanced Microscopie}

% Fußzeile
\addbibresource{AdvancedMicroscopy.bib}

\usepackage{luacode}

\DeclareSIUnit\px{px}
\DeclareSIUnit\strich{|||}
\DeclareSIUnit\Var{var}
\DeclareSIUnit\VA{VA}
\DeclareSIUnit\bar{bar}

\usepackage{cleveref}

\crefname{enumerate}{Aufzählung}{Aufzählungen}

\begin{document}

\begin{luacode*}
dofile("createExtraPDF.lua")
\end{luacode*}

% \includepdf{./deckblatt.pdf}
\tableofcontents

\newpage

\section{Aufgabenstellung}\label{Auf}

\subsection{Abbildung durch eine Sammellinse}

\begin{itemize}
	\item Die Brennweite ist durch den ``Lupeneffekt'' abzuschätzen
	\item Die Brennweite der Linse ist in einem dafür aufzubauenden Strahlengang mit
	      Hilfe der Abbildungsgleichung zu bestimmen
	\item Die Brennweite der Linse ist nach dem Bessel-Verfahren zu berechnen
	\item Der Abbildungsmaßstab ist auf $A$ = 1 einzustellen
	\item Die Ergebnisse sind miteinander zu vergleichen und zu diskutieren
\end{itemize}

\subsection{Das 1-Linsen-Mikroskop nach van Leeuwenhoek}

\begin{itemize}
	\item Das 1-Linsen-Mikroskop ist aufzubauen
	\item Ein Objekt ist mit dem 1-Linsen-Mikroskop nach van Leeuwenhoek abzubilden
	\item Die Fokuslänge und die Vergrößerung ist zu bestimmen
\end{itemize}

\subsubsection{Hellfeld-Transmissionsmikroskop}

\begin{itemize}
	\item Aufbau des Hellfeld-Transmissionsmikroskops
	\item Ausrichten der Linsen um ein scharfes Bild zu erzeugen
	\item Linsenorientierung (Objektiv) und sphärische Aberration: Drehen der
	      Objektiv-Linse (mitsamt Halterung) um 180° und Einbau dieser
	\item Kohärenz der Beleuchtung: Entfernung des Diffusors
	\item Wellenlängenabhängigkeit und chromatische Aberration: Verwendung von
	      verschiedenfarbigen LEDs als Lichtquelle
	\item Messung der experimentellen Gesamtvergrößerung
	\item Bestimmung des Auflösevermögens durch Betrachtung der entsprechenden
	      Balkenpaare
\end{itemize}

\subsubsection{Aufbau eines einfachen Dunkelfeld-Mikroskops}

\begin{itemize}
	\item Aufbau des einfachen Dunkelfeld-Mikroskops
	\item Vergleich der gemessenen Bilder mit jenen, des zuvor aufgebauten
	      Hellfeld-Transmissionsmikroskops
	\item Überlegen und skizzieren eines Dunkelfeld-Aufbaus in einer Auflicht-Variante
\end{itemize}

\section{Grundlagen}\label{Grund}

\subsection{Abbildung durch eine Sammellinse}

Vergrößerungsgläser bündeln die Lichtstrahlen und bringen diese näher an das
Auge heran, um die Nahsicht zu verbessern. Es handelt sich um konvexe Gläser,
das bedeutet, sie sind in der Mitte dicker und an den Rändern dünner. Wenn
Licht durch eine konvexe Linse fällt, wird es gebrochen, also der Lichtstrahl
nach innen abgelenkt, wie in \autoref{fig:strahlengang_linse} sichtbar. Die
Lichtstrahlen laufen in einem Punkt auf der anderen Seite der Linse zusammen.
Dieser Punkt wird als Brennpunkt bezeichnet.

\begin{figure}[H]
	\begin{center}
		% \includegraphics[width =0.8\textwidth]{./figures/skizze_linsengleichung.JPG}
	\end{center}
	\caption[Skizze des Strahlengangs durch eine Sammellinse] { Skizze des Strahlengangs
		durch eine Sammellinse \cite{banzer_advanced_2022} \\
		$G$ \dots Gegenstand                               \\
		$g$ \dots Gegenstandsweite                         \\
		$B$ \dots Bild                                     \\
		$b$ \dots Bildweite                                \\
		$f$ \dots Brennweite
	}\label{fig:strahlengang_linse}
\end{figure}

\subsubsection{Lupeneffekt}

Um die Brennweite einer Konvexlinse mit einem Objekt im Unendlichen zu
ermitteln, kann folgende Formel verwendet werden:

\begin{equation}
	\frac{1}{f} = \lim_{g\to\infty}\frac{1}{g} +\frac{1}{b} = \frac{1}{b}
	\label{eq:herleitungAbbildungUnendlich}
\end{equation}

wobei $f$ die Brennweite und $b$ der Abstand zwischen Objektiv und Bild ist.
Wenn sich das Objekt im Unendlichen befindet, ist der Abstand $b$ gleich der
Brennweite $f$, so dass sich die Formel vereinfacht zu:

\begin{equation}
	f = b
	\label{eq:AbbildungUnendlich}
\end{equation}

Das bedeutet, dass die Brennweite eines Objektivs durch Messen des Abstands
zwischen dem Objektiv und dem Bild bestimmt werden kann, wenn sich das Objekt
im Unendlichen befindet.

\subsubsection{Abbildungsgleichung}
Die Formel für dünne Linsen kann verwendet werden, um die Brennweite einer
konvexen Linse zu berechnen, wenn sich das Objekt nicht im Unendlichen
befindet. Die Formel lautet:

\begin{equation}
	\frac{1}{f} = \frac{1}{g} +\frac{1}{b}
	\label{eq:linsengleichung}
\end{equation}

Dabei ist $f$ die Brennweite, $g$ ist der Abstand zwischen Objekt und Linse und
$b$ ist der Abstand zwischen Linse und Bild. Durch Messung der Abstände $g$ und
$b$ kann die Brennweite mit dieser Formel berechnet werden.

\subsubsection{Bessel Verfahren}
Die Bessel-Methode ist ein weiteres Verfahren zur Bestimmung der Brennweite
eines Objektivs. Dabei wird eine Lichtquelle in einem festen Abstand von einer
konvexen Linse platziert und die Linse hin- und herbewegt, bis zwei Positionen
gefunden werden, in denen das Bild der Lichtquelle scharf ist. Der Abstand
zwischen diesen beiden Positionen wird dann in eine Formel zur Berechnung der
Brennweite der Linse eingesetzt. Die Formel für die Bessel-Methode lautet:

\begin{equation}
	f = \frac{l^2-w^2}{4l}
	\label{eq:bessel}
\end{equation}

Dabei ist $f$ die Brennweite, $l$ der Abstand zwischen den beiden Positionen,
an denen das Bild scharf abgebildet wird, und $w$ der Abstand zwischen der
Lichtquelle und dem Objektiv. Diese Methode ist präziser als die Formel für
dünne Linsen und kann für Objektive mit einer großen Bandbreite an Brennweiten
% verwendet werden.~\cite{banzer_advanced_2022} %todo \cite

\subsection{Das 1-Linsen-Mikroskop nach van Leeuwenhoek}
Das Mikroskop von van Leeuwenhoek, auch Einlinsen- oder einfaches Mikroskop
genannt, funktioniert mit einer kleinen, kugelförmigen Linse, die das Bild
eines Objekts vergrößert. Die Linse ist in einem kleinen Halter mit einer
Lochblende in der Mitte montiert. Das zu betrachtende Objekt wird sehr nahe an
die Linse gehalten, und der Beobachter schaut durch die Lochblende auf das
vergrößerte Bild. Die Linse und das Objekt werden zusammen bewegt, bis das Bild
scharf ist.

\begin{figure}[H]
	\begin{center}
		% \includegraphics[width =0.5\textwidth]{./figures/leeuwenhoek.JPG}
	\end{center}
	\caption[1-Linsen-Mikroskop nach van Leeuwenhoek] { 1-Linsen-Mikroskop nach van
		% Leeuwenhoek \cite{banzer_advanced_2022}
	}\label{fig:leeuenhoek}
\end{figure}

Mit der Kugellinsengleichung lässt sich die Brennweite der Linse des van
Leeuwenhoek-Mikroskops berechnen. Die Kugellinsengleichung lautet:

\begin{equation}
	f_{eff}= \frac{nd}{4(n-1)}
	\label{eq:effektive_fokuslange}
\end{equation}

Dabei ist $f$ die Brennweite der Linse, $n$ der Brechungsindex des Materials,
aus dem die Linse besteht, und $d$ der Durchmesser der Kugellinse.

Um die Vergrößerung des van Leeuwenhoek-Mikroskops zu berechnen, muss man die
Bezugssehweite des Auges des Beobachters (\SI{250}{\mm}) kennen. Die
Vergrößerung wird dann nach folgender Formel berechnet:

\begin{equation}
	M = \frac{f_\text{Bezugssehweite}}{f}= \frac{\SI{250}{\mm}}{f}
	\label{eq:LeeuwenhoekVergroeserung}
\end{equation}

wobei $f$ die effektive Brennweite der Kugellinse und $f_\text{Bezugssehweite}$
die typische Bezugssehweite des Menschen (\SI{250}{\mm})
% ist.~\cite{kuchling_taschenbuch_2014}

\subsection{Hellfeld-Transmissionsmikroskop}
Die Hellfeld-Transmissionsmikroskopie ist eine Technik zur Beobachtung der
Struktur von transparenten oder durchscheinenden Proben wie Zellen oder dünnen
Gewebeschnitten. Dazu wird ein Lichtstrahl durch die Probe und dann durch eine
Reihe von Linsen geleitet, die das Licht vergrößern und fokussieren, um ein
Bild zu erzeugen.

Ein einfaches Hellfeld-Transmissionsmikroskop kann mit zwei Linsen gebaut
werden, einer Objektivlinse und einer Okularlinse. Die Objektivlinse befindet
sich am nächsten zur Probe und ist für die Sammlung und Vergrößerung des durch
sie einfallenden Lichts verantwortlich. Die Okularlinse befindet sich näher am
Auge des Betrachters und dient dazu, das von der Objektivlinse erzeugte Bild
weiter zu vergrößern.

\begin{figure}[H]
	\begin{center}
		% \includegraphics[width =\textwidth]{./figures/skizze_hellfeldmikroskop.JPG}
	\end{center}
	\caption[Skizze des Strahlengangs durch ein Hellfeld-Transmissionsmikroskop] { Skizze
		des Strahlengangs durch ein Hellfeld-Transmissionsmikroskop \\
		% \cite{banzer_advanced_2022}                   \\
		$G$ \dots Gegenstand                                        \\
		$B_{ZB}$ \dots Zwischenbild                                 \\
		$f_{OBJ}$ \dots Brennweite der Objektiv-Linse               \\
		$f_{OK}$ \dots Brennweite der Okular-Linse
	}\label{fig:strahlengang_hellfeldmikroskop}
\end{figure}

Der Strahlengang eines Mikroskops sieht folgendermaßen aus:

\begin{enumerate}
	\item Die Lichtquelle erzeugt einen Lichtstrahl, der durch eine Kondensorlinse fällt,
	      die das Licht auf die Probe fokussiert.
	\item Das Licht durchdringt die Probe, die je nach ihren Eigenschaften einen Teil des
	      Lichts absorbieren oder streuen kann.
	\item Die Objektivlinse sammelt das Licht, das durch die Probe fällt, und erzeugt ein
	      vergrößertes Bild der Probe.
	\item Die Okularlinse vergrößert das von der Objektivlinse erzeugte Bild weiter und
	      projiziert es auf das Auge des Betrachters.
	\item Der Beobachter sieht ein vergrößertes und fokussiertes Bild der Probe.
\end{enumerate}

Um das Auflösevermögen des Mikroskops zu bestimmen, wird eine spezielle Probe
verwendet. Auf dieser Probe befinden sich verschiedene Balkenanordnungen, die
in Gruppen jeweils vertikal und horizontal angeordnet sind. Jeder dieser
Gruppen ist eine Nummer zugeordnet. Aufgrund der Anordnung dieser Balken und
der entsprechenden Nummer kann das Auflösevermögen nach folgender Formel
% bestimmt werden~\cite{banzer_advanced_2022}:

\begin{equation}
	\text{Balkenpaare} / \text{mm} = 2^{\text{Gruppe} + \frac{\text{Element}-1}{6}}
	\label{eq:auflosungsvermogen}
\end{equation}

\subsection{Aufbau eines einfachen Dunkelfeld-Mikroskops}
Die Dunkelfeld-Mikroskopie ist eine Technik zur Untersuchung von Proben, die
entweder sehr klein oder sehr dünn sind und keinen ausreichenden Kontrast für
die Hellfeldmikroskopie bieten. Bei dieser Technik wird eine ringförmige Blende
verwendet, die das direkt übertragene Licht blockiert. So gelangt nur das
Streulicht in den Aufbau. So entsteht ein helles Bild des Objekts vor einem
dunklen Hintergrund, daher der Name "Dunkelfeld".

Um ein Dunkelfeldmikroskop mit zwei Objektiven zu erstellen, wird ein
spezielles Kondensorobjektiv zum Aufbau eines Hellfeldmikroskops hinzugefügt.
Das Kondensorobjektiv ist mit einer lichtundurchlässigen Scheibe mit einer
kleinen Öffnung in der Mitte ausgestattet, die direktes Licht blockiert und nur
Streulicht durchlässt.

\begin{figure}[H]
	\begin{center}
		% \includegraphics[width =\textwidth]{./figures/skizze_dunkelfeldmikroskop.JPG}
	\end{center}
	\caption[Vergleich zwischen Hellfeld- (links) und Dunkelfeld-Methode (rechts)] {
		Vergleich zwischen Hellfeld- (links) und Dunkelfeld-Methode (rechts)
		% \cite{banzer_advanced_2022}
	}\label{fig:strahlengang_dunkelfeldmikroskop}
\end{figure}

Der Strahlengang des Mikroskops sieht folgendermaßen aus:

\begin{enumerate}
	\item Eine Lichtquelle erzeugt einen Lichtstrahl, der durch den Dunkelfeldkondensor
	      auf die Probe fokussiert wird.
	\item Die Probe streut das Licht, und nur das gestreute Licht gelangt durch die
	      ringförmige Blende der Kondensorlinse.
	\item Das gestreute Licht wird von der Objektivlinse aufgefangen und bildet ein
	      helles Bild der Probe vor einem dunklen Hintergrund.
	\item Das Bild wird durch die Okularlinse weiter vergrößert und kann vom Beobachter
	      betrachtet werden.
\end{enumerate}

Der dunkle Hintergrund im Bild entsteht durch das Fehlen von direktem Licht,
das durch die ringförmige Blende des Kondensors blockiert worden wäre. Aufgrund
der Abhängigkeit vom Brechungsindex und Streueigenschaften können kleine,
transparente oder durchscheinende Proben wie Bakterien oder kleine Partikel
sichtbar gemacht werden, die mit anderen Mikroskopietechniken nur schwer zu
erkennen wären, oder erst speziell eingefärbt werden
% müssten.~\cite{banzer_advanced_2022}

\section{Versuchsanordnung}\label{sec:versuchsanordnung}

\subsection{Abbildung durch eine Sammellinse}

Zunächst wird die Brennweite der Linse mit dem Lupeneffekt abgeschätzt. Dazu
wird diese in der Hand gehalten, um ein Bild abzubilden. Das genauere Vorgehen
wird in \autoref{sec:versuchsdurchfuehrung_messergebnisse} erläutert.

Der Aufbau für den restlichen Versuch ist in \autoref{fig:aufbau_linse}
sichtbar. Um die Bildweite bequem messen und das Bild entsprechend scharf
stellen zu können, wird nach Rücksprache mit dem Betreuer ein Schirm aus
Komponenten aus dem Lagerbestand und einem Stück Papier gebaut, wie in
\autoref{fig:aufbau_linse} sichtbar. Um den Gegenstand abbilden zu können, wird
dieser mit einer Halogenlampe ausgeleuchtet.

\begin{figure}[H]
	\begin{center}
		% \includegraphics[width =0.5\textwidth]{./figures/aufbau_linse.JPG}
	\end{center}
	\caption[Versuchsaufbau zur Bestimmung der Brennweite einer Sammellinse] {
		Versuchsaufbau zur Bestimmung der Brennweite einer Sammellinse \\
		1 \dots Linse, deren Brennpunkt zu bestimmen ist               \\
		2 \dots Schirm                                                 \\
		3 \dots Halterung mit betrachteten Gegenstand
	}\label{fig:aufbau_linse}
\end{figure}

\subsection{Das 1-Linsen-Mikroskop nach van Leeuwenhoek}

Das 1-Linsen-Mikroskop nach van Leeuwenhoek wird zunächst aus den Teilen,
sichtbar in \autoref{fig:mikroskop_bauteile}, zusammengebaut. Zunächst werden
die Kugellinsen in die entsprechenden Vertiefungen des Plastiksteifens
gesteckt. Nun wird dieser mithilfe der Schraube und Mutter an der Grundplatte
befestigt. In diese wird in die entsprechende Versenkung eine Knopfbatterie und
darunter ein Fuß der LED geklemmt. Der zweite Fuß wird so verbogen, dass er
sich über der zweiten Seite der Batterie befindet, diese aber nicht berührt.
Nun wird eine der beiden Büroklammern leicht aufgebogen, durch das Loch in der
Grundplatte gefädelt und über den Kontakt der LED geführt, dass das Berühren
dieser wie ein Schalter für die LED funktioniert. Die zweite Büroklammer wird
durch das Loch auf der anderen Seite der Grundplatte gefädelt. Diese dient
dazu, die andere Büroklammer nach unten zu spannen, wenn eine Dauerbeleuchtung
gewünscht ist. Das fertige Modell ist in \autoref{fig:mikroskop_fertig}
sichtbar.

\begin{figure}[H]
	\centering
	\begin{subfigure}{.30\linewidth}
		% \includegraphics[width=\textwidth]{./figures/mikroskop_bauteile.JPG}
		\caption{Verwendete Bauteile
		}\label{fig:mikroskop_bauteile}%\cite{banzer_advanced_2022}
	\end{subfigure}
	\begin{subfigure}{.50\linewidth}
		% \includegraphics[width=\textwidth]{./figures/mikroskop.png}
		\caption{Fertiges Modell
		}\label{fig:mikroskop_fertig}
	\end{subfigure}
	\caption[1-Linsen-Mikroskop nach van Leeuwenhoek] {1-Linsen-Mikroskop nach van
		Leeuwenhoek
	}\label{fig:leeuwenhoek_mikroskop}
\end{figure}

\newpage
\subsection{Hellfeld-Transmissionsmikroskop}

Der Versuchsaufbau für das Hellfeld-Transmissionsmikroskop ist in
\autoref{fig:aufbau_hellmikroskopie} ersichtlich. Es sind folgende Schritte
durchzuführen. Die Nummer des Schritts entspricht dabei der Nummer des Bauteils
in \autoref{fig:aufbau_hellmikroskopie}.

\begin{enumerate}
	\item Zunächst wird die Halogenlampe auf der Aufbauschiene befestigt und über das
	      entsprechende Kabel eine Stromversorgung hergestellt.
	\item Nun wird sichergestellt, dass diffuses Licht in den Aufbau gelangt, indem der
	      Diffusor eingebaut wird. Dieser besteht grundsätzlich aus 2 kreuzweise
	      angeordneten Klebebandstreifen.
	\item Hier wird der Probenhalter befestigt in dem die Glasprobe mithilfe von 2
	      Madenschrauben befestigt wird. Hierbei ist besondere Vorsicht geboten, um den
	      Körper nicht zu beschädigen. Die genaue Position, die vom Aufbau durchstrahlt
	      wird, kann mithilfe der Justierungs-Schrauben verändert werden.
	\item Hinter dem Gegenstand wird nun die Obektiv-Linse mit einer Brennweite von
	      \SI{35}{\milli\meter} plaziert.
	\item Nun wird die Okular-Linse mit einer Brennweite von \SI{50}{\milli\meter} so in
	      den Aufbau gegeben, dass sich der Brennpunkt dieser dort befindet, wo zuvor das
	      Bild war. Dadurch wird sichergestellt, dass der Strahlengang dahinter parallel
	      verläuft und quasi im Unendlichen fokusiert ist, was für eine Abbildung der
	      Kamera notwendig ist.
	\item Am anderen Ende der Schiene wird nun die Kamera positioniert. Vor dieser
	      befindet sich auch eine Linse, um das Bild auf die Kamera abbilden zu können.
\end{enumerate}

\begin{figure}[H]
	\begin{center}
		% \includegraphics[width =\textwidth]{./figures/aufbau_hellmikroskopie.jpg}
	\end{center}
	\caption[Versuchsaufbau zur Bestimmung der Brennweite einer Sammellinse] {
		Versuchsaufbau zur Bestimmung der Brennweite einer Sammellinse        \\
		1 \dots Halogenlampe mit IR-Filter                                    \\
		2 \dots Diffusor                                                      \\
		3 \dots Halterung mit betrachteten Gegenstand                         \\
		4 \dots Objektiv-Linse mit einer Brennweite von \SI{35}{\milli\meter} \\
		5 \dots Okular-Linse mit einer Brennweite von \SI{50}{\milli\meter}   \\
		6 \dots Kamera mit einer Linse mit einer Brennweite von \SI{150}{\milli\meter}
	}\label{fig:aufbau_hellmikroskopie}
\end{figure}

\subsection{Aufbau eines einfachen Dunkelfeld-Mikroskops}

Der Versuchsaufbau für das ein einfaches Dunkelfeld-Mikroskops ist in
\autoref{fig:aufbau_dunkelfeldmikroskopie} ersichtlich. Es sind folgende
Schritte durchzuführen. Die Nummer des Schritts entspricht dabei der Nummer des
Bauteils in \autoref{fig:aufbau_dunkelfeldmikroskopie}. Bei diesen Versuch ist
es besonders wichtig darauf zu achten, dass die Höhen der einzelnen Komponenten
auf einer Ebene ausgerichtet sind.

\begin{enumerate}
	\item Zunächst wird die Halogenlampe auf der Aufbauschiene befestigt und über das
	      entsprechende Kabel eine Stromversorgung hergestellt.
	\item Hinter die Lichtquelle wird das Nikon-Objektiv eingesetzt, um für eine
	      Beleuchtung der Probe zu sorgen. In diese ist bereits eine Ringblende
	      integriert, um nur Randstrahlen durchzulassen. Die korrekte Ausrichtung dieser
	      kann überprüft werden, indem der Ring dahinter mit einem Stück Papier
	      betrachtet wird. Dieser sollte gleichmäßig ausgeleuchtet erscheinen.
	\item Hinter dem Objektiv wird nun der Probenhalter mit der eingeschraubten Glasprobe
	      Befestigt. Der Abstand zwischen Objektiv und Probe sollte hierbei
	      \SI{6.2}{\milli\meter} betragen.
	\item Hier wird eine Kombination aus achromatischer Linse und Blende in den
	      Strahlengang gegeben. Die Blende dient dazu die Beleuchtung auszublenden,
	      sodass nur die Dunkelfeld-Strahlen erfasst werden. Die genaue Position der
	      Linse muss im fertigen Aufbau variiert werden, bis ein Dunkelfeldbild entsteht.
	\item Um ein Bild aufnehmen zu können, wird die Kamera in den Aufbau gegeben. Diese
	      wird ohne eine zusätzliche Linse betrieben.
	\item Im finalen Aufbau müssen die Positionen der Kamera und der Linse verändert
	      werden, bis ein Dunkelfeld Bild entsteht. Für die Distanz zwischen Kamera und
	      Linse hat sich genau der Abstand von 3 Sockeln als ideal erwiesen, um ein
	      scharfes Bild erzeugen zu können. Aus diesem Grund wurden die Sockel auf die
	      Schiene gegeben und im weiteren Verlauf immer das gesamte System aus Linse,
	      Sockel und Kamera verschoben.
\end{enumerate}

\begin{figure}[H]
	\begin{center}
		% \includegraphics[width =0.5\textwidth]{./figures/aufbau_dunkelfeldmikroskopie.jpg}
	\end{center}
	\caption[Versuchsaufbau zur Bestimmung der Brennweite einer Sammellinse] {
		Versuchsaufbau zur Bestimmung der Brennweite einer Sammellinse \\
		1 \dots Halogenlampe mit IR-Filter                             \\
		2 \dots Mikroskopobjektiv mit integrierter Dunkelfeld-Blende   \\
		3 \dots Halterung mit betrachteten Gegenstand                  \\
		4 \dots achromatische Linse mit einer Brennweite von \SI{25}{\milli\meter} mit
		verstellbarer Blende                                           \\
		5 \dots Kamera ohne Linse                                      \\
		6 \dots verwendete Sockel, um die Distanz zu fixieren
	}\label{fig:aufbau_dunkelfeldmikroskopie}
\end{figure}

\section{Geräteliste}\label{sec:geraeteliste}

\subsection{Abbildung durch eine Sammellinse}

Für die Abbildung durch eine Sammellinse werden die in
\autoref{tab:gerate_linse} aufgelisteten Geräte verwendet.

\begin{table}[H]
	\begin{center}
		\caption{Verwendete Geräte für die Abbildung durch eine Sammellinse
		}
		\begin{tblr}{cells={font=\footnotesize},colspec={lllll}}
			\textbf{Gerätetyp}               & \textbf{Hersteller} & \textbf{Typ} & \textbf{Anmerkung} \\
			Linse                            &                     &              & zu bestimmen       \\
			Lens Mount                       & ThorLabs            & LMR1/M       &                    \\
			Optical Posts                    & ThorLabs            & TR3          &                    \\
			Rail Carrier                     & ThorLabs            & XT34TR1/M    & 3x                 \\
			Halogenlampe                     & ThorLabs            & QTH10/M      &                    \\
			Mount for Rectangular Optics     & ThorLabs            & XYF1/M       &                    \\
			Resolution and Distortion Target & ThorLab             & R1L3S5P      &                    \\
			Aluminium-Schiene                &                     &              &                    \\
			Schirm                           &                     &              & selbst gebaut      \\
			Schibelehre                      & Workzone            & 819547       & digital            \\
		\end{tblr}\label{tab:gerate_linse}
	\end{center}
\end{table}

\newpage
\subsection{Das 1-Linsen-Mikroskop nach van Leeuwenhoek}

Für den Aufbau des 1-Linsen-Mikroskop nach van Leeuwenhoek werden die in
\autoref{tab:gerate_leeuenhoek} aufgelisteten Teile benötigt.

\begin{table}[H]
	\begin{center}
		\caption{Verwendete Geräte für den Zusammenbau des 1-Linsen-Mikroskop nach van Leeuwenhoek
		}
		\begin{tblr}{cells={font=\footnotesize},colspec={lllll}}
			\textbf{Bauteil}     & \textbf{Anmerkung}                  \\
			Glaskugel            & Durchmesser \SI{2.5}{\milli\meter}  \\
			Glaskugel            & Durchmesser \SI{6.35}{\milli\meter} \\
			Halterung für Linsen & 3D-gedruckt                         \\
			LED                  & weiß                                \\
			Knopfbatterie        & CR2032                              \\
			Büroklammern         & 2x                                  \\
			Trägerplatte         & 3D-gedruckt                         \\
			Schraube             &                                     \\
			Mutter               &
		\end{tblr}\label{tab:gerate_leeuenhoek}
	\end{center}
\end{table}

\newpage
\subsection{Hellfeld-Transmissionsmikroskop}

Für den Aufbau des Hellfeld-Transmissionsmikroskops werden die in
\autoref{tab:gerate_hellfeldmikroskop} aufgelisteten Geräte verwendet.

\begin{table}[H]
	\begin{center}
		\caption{Verwendete Geräte für das Hellfeld-Transmissionsmikroskop
		}
		\begin{tblr}{cells={font=\footnotesize},colspec={lllll}}
			\textbf{Gerätetyp}               & \textbf{Hersteller} & \textbf{Typ} & \textbf{Anmerkung}         \\
			Halogenlampe                     & ThorLabs            & QTH10/M      & mit IR-Filter              \\
			Diffusor auf Slip Ring           & ThorLabs            & SM1RC/M      &                            \\
			Mount for Rectangular Optics     & ThorLabs            & XYF1/M       &                            \\
			Resolution and Distortion Target & ThorLab             & R1L3S5P      &                            \\
			Objektiv-Linse                   &                     &              & F = \SI{35}{\milli\meter}  \\
			Lens Mount                       & ThorLabs            & LMR1/M       &                            \\
			Okular-Linse                     & ThorLabs            & SM1V10       & F = \SI{50}{\milli\meter}  \\
			Lens Mount                       & ThorLabs            & SM1RC/M      &                            \\
			Blende                           & ThorLabs            & SM2D25       &                            \\
			Kamera                           & ThorLabs            & CS165MU/M    &                            \\
			Kamera-Linse                     &                     &              & F = \SI{150}{\milli\meter} \\
			Optical Posts                    & ThorLabs            & TR3          & 5x                         \\
			Rail Carrier                     & ThorLabs            & XT34TR1/M    & 6x                         \\
			Aluminium-Schiene                &                     &              &                            \\
			Schibelehre                      & Workzone            & 819547       & digital                    \\
			Computersoftware                 & ThorLabs            & ThorCam      &
		\end{tblr}\label{tab:gerate_hellfeldmikroskop}
	\end{center}
\end{table}

\newpage
\subsection{Aufbau eines einfachen Dunkelfeld-Mikroskops}

Für den Aufbau des einfachen Dunkelfeld-Mikroskops werden die in
\autoref{tab:gerate_dunkelfeldmikroskop} aufgelisteten Geräte verwendet.

\begin{table}[H]
	\begin{center}
		\caption{Verwendete Geräte für das einfachen Dunkelfeld-Mikroskop
		}
		\begin{tblr}{cells={font=\footnotesize},colspec={lllll}}
			\textbf{Gerätetyp}               & \textbf{Hersteller} & \textbf{Typ} & \textbf{Anmerkung}        \\
			Halogenlampe                     & ThorLabs            & QTH10/M      & mit IR-Filter             \\
			Mikroskopobjektiv                & Nikon               & MRP40102     & mit Dunkelfeldblende      \\
			Mount for Rectangular Optics     & ThorLabs            & XYF1/M       &                           \\
			Resolution and Distortion Target & ThorLab             & R1L3S5P      &                           \\
			Achomatische Linse               &                     &              & F = \SI{25}{\milli\meter} \\
			Blende                           & ThorLabs            & SM1D12C      &                           \\
			Lens Mount                       & ThorLabs            & SM1RC/M      &                           \\
			Kamera                           & ThorLabs            & CS165MU/M    &                           \\
			Optical Posts                    & ThorLabs            & TR3          & 4x                        \\
			Rail Carrier                     & ThorLabs            & XT34TR1/M    & 8x                        \\
			Aluminium-Schiene                &                     &              &                           \\
			Schibelehre                      & Workzone            & 819547       & digital                   \\
			Computersoftware                 & ThorLabs            & ThorCam      &
		\end{tblr}\label{tab:gerate_dunkelfeldmikroskop}
	\end{center}
\end{table}

\section{Versuchsdurchführung und Messergebnisse}\label{sec:versuchsdurchfuehrung_messergebnisse}

\subsection{Abbildung durch eine Sammellinse}

\subsubsection{Lupeneffekt}

Um die Brennweite durch den Lupeneffekt zu bestimmen, wird die Linse vor den
Schirm gehalten und die Distanz so lange variiert, bis am Schirm ein scharfes
Bild entsteht. Als Schirm wird dabei ein Kasten gewählt, um die Schiebelehre
anlegen zu können und so die Distanz genau bestimmen zu können. Als Bild wurde
dabei die blaue Plane am Dach des Nebengebäudes gewählt, weil dies einem
unendlich weit entfernten Bild am ehesten entspricht. Die blaue Plane wurde
dabei deshalb gewählt, weil diese aufgrund ihrer Farbe leicht identifiziert
werden konnte und so problemlos scharf gestellt werden konnte. Die
entsprechende Durchführung ist dabei in \autoref{fig:lupeneffekt} ersichtlich.

\begin{figure}[H]
	\begin{center}
		% \includegraphics[width =0.5\textwidth]{./figures/lupeneffekt.JPG}
	\end{center}
	\caption[Abbildung auf Schirm durch die Linse] { Abbildung auf Schirm durch die Linse
	}\label{fig:lupeneffekt}
\end{figure}

Durch den Aufbau entspricht nun die Bildweite $b'$ dem Abstand des Brennpunkts.
Dieser wird mithilfe der Schiebelehre bestimmt, was folgenden Wert liefert:

\begin{equation*}
	b' = \SI{24(3)}{\mm}
\end{equation*}

Die Unsicherheit wurde dabei entsprechend groß gewählt, da sich die Distanz aus
der subjektiven Wahrnehmung der Schärfe ergibt.

Von der so bestimmten Distanz muss noch die halbe Dicke der Linse hinzugefügt
werden, die im konkreten Fall $d_\text{Lupe} =\SI{13(5)}{\mm}$ entspricht.

\subsubsection{Abbildungsgleichung}

Um die Brennweite der Linse mithilfe der Abbildungsgleichung zu bestimmen, wird
der Versuch zunächst nach \autoref{fig:aufbau_linse} aufgebaut. Die Position
des Schirms muss dabei so lange variiert werden, bis eine scharfe Abbildung des
Gegenstands entsteht. Nun werden die Distanzen für die Gegenstands und
Bildweite mithilfe der Schiebelehre bestimmt. Dies wird für 2 verschiedene
Distanzen durchgeführt. Die erhaltenen Ergebnisse sind dabei in
\autoref{tab:werte_linsen} aufgelistet.

\begin{table}[H]
	\caption[Gemessene Bild und Gegenstandsweiten zur Bestimmung mithilfe der
		Abbildungsgleichung] {Gemessene Bild und Gegenstandsweiten zur Bestimmung
		mithilfe der Abbildungsgleichung          \\
		$b$ \dots gemessene Bildweite in \si{\mm} \\
		$g$ \dots gemessene Bildweite in \si{\mm}
	}
	\label{tab:werte_linsen}
	\centering
	\begin{tblr}{SS}
		{{{$b$ / \si{\mm}}}} & {{{$g$ / \si{\mm}}}} \\
		49(2)                & 44.4(5)              \\
		68(2)                & 32.3(5)              \\
	\end{tblr}
\end{table}

Zudem wurde die Sockeldicke gemessen
$d_\text{Sockel}=\SI{15.37(2)}{\milli\meter}$ gemessen.

\subsubsection{Bessel-Verfahren}

Um die Brennweite anhand des Bessel-Verfahrens zu bestimmen, wird der gleiche
Versuchsaufbau wie für die Abbildungsgleichung verwendet. Der Unterschied hier
ist, dass die Positionen des Schirms und des Gegenstands nicht verändert
werden. Es wird nur die Linse verschoben, um die beiden Positionen zu finden,
an denen das Bild als scharf erscheint. Die abgemessenen Werte sind in
\autoref{tab:werte_bessel} sichtbar. Die gesamte Länge wird dabei ebenfalls
gemessen, was einen Wert von $l=\SI{116.29(2)}{\mm}$ liefert.

\begin{table}[H]
	\centering
	\caption[Gemessene Bild und Gegenstandsweiten zur Bestimmung mithilfe des
		Besselverfahrens]{ Gemessene Bild und Gegenstandsweiten zur Bestimmung mithilfe
		des Besselverfahrens. Die Länge dazwischen beträgt dabei
		$l=\SI{116.29(2)}{\mm}$                   \\
		$b$ \dots gemessene Bildweite in \si{\mm} \\
		$g$ \dots gemessene Bildweite in \si{\mm}
	} \label{tab:werte_bessel}
	\begin{tblr}{SS}
		{{{$b$ / \si{\mm}}}} & {{{$g$ / \si{\mm}}}} \\
		68(2)                & 32.3(5)              \\
		39.72(2)             & 61.38(2)             \\
	\end{tblr}
\end{table}

\subsubsection{Abbildungsmaßstab}

Um ein Bild mit dem Abbildunggsmaßstab $A$ = 1 aufzunehmen, wird der Schirm
durch die Kamera ausgetauscht. Um mit der Kamera ein Bild aufnehmen zu können,
wird diese über die USB-Verbindung mit dem Computer verbunden. Auf diesem wird
die Software ``ThorCam'' geöffnet und zunächst die Kamera verbunden. Über den
grünen ``Play-Knopf'' kann die Übertragung des Bilds gestartet werden.
Zusätzlich muss in den Einstellungen ``Continuous Auto-Scale'' und
``Auto-Exposure'' ausgewählt werden. Beim Sichern der Bilder ist wichtig, dass
diese als ``8-bit JPEG with Annotations'' gespeichert werden.

Um einen Abbildungsmaßstab von 1 zu garantieren ist wichtig darauf zu achten,
dass die Bildweite der Gegenstandsweite entspricht. Das so erzeugte Bild ist in
\autoref{fig:abbildungsmasstab} sichtbar.

\begin{figure}[H]
	\begin{center}
		% \includegraphics[width =0.5\textwidth]{./figures/advmicroskope/linsebrennweitenbestimmungScharf.jpg}
	\end{center}
	\caption[Erzeugte Abbildung für einen Abbildunggsmaßstab von 1] { Erzeugte Abbildung für
		einen Abbildunggsmaßstab von 1
	}\label{fig:abbildungsmasstab}
\end{figure}

\subsection{Das 1-Linsen-Mikroskop nach van Leeuwenhoek}

Mit dem selbstgebauten 1-Linsen-Mikroskop nach van Leeuwenhoek werden nun
verschiedene Proben betrachtet. So wird ein kleingedruckter Schriftzug
vergrößert und die Struktur und Gravur auf einem Taschenmesser betrachtet, wie
in \autoref{fig:van_leeuwenhoek_bild} sichtbar. Dabei ist vor allem auf eine
geeignete Beleuchtung zu achten, da die Position des LEDs nicht ideal gewählt
ist. Es sei angemerkt, dass es sehr schwer ist ein sauberes Bild davon
aufzunehmen.

\begin{figure}[H]
	\begin{center}
		% \includegraphics[width =0.5\textwidth]{./figures/mikroskop_bild.JPG}
	\end{center}
	\caption[Erzeugtes Bild mit 1-Linsen-Mikroskop nach van Leeuwenhoek] { Erzeugtes Bild
		mit 1-Linsen-Mikroskop nach van Leeuwenhoek
	}\label{fig:van_leeuwenhoek_bild}
\end{figure}

\subsection{Hellfeld-Transmissionsmikroskop}

Zunächst muss das Hellfeld-Transmissionsmikroskop, wie bereits in
\autoref{sec:versuchsanordnung} erklärt, aufgebaut werden. Wenn durch die
Positionen der Linsen sichergestellt ist, dass das erzeugte Bild scharf ist und
alle erforderlichen Einstellungen durchgeführt wurden, kann mit den
eigentlichen Aufgaben begonnen werden.

\subsubsection{Linsenorientierung und sphärische Aberration}

Zunächst wird die Linsenorientierung umgekehrt. Dies wird erreicht, indem die
Linse samt Halterung um 180° gedreht wird. Das bedeutet, dass nun die gekrümmte
Seite in die Richtung der Probe weist. Durch die so erzeugte Änderung wird das
entsprechende Bild aufgenommen, was in \autoref{fig:hell_umdrehen} sichtbar
ist.

\begin{figure}[H]
	\begin{center}
		% \includegraphics[width =0.5\textwidth]{./figures/advmicroskope/hellmikro-objetive180.jpg}
	\end{center}
	\caption[Unscharfes Bild durch das Umdrehen der Linse] { Unscharfes Bild durch das
		Umdrehen der Linse
	}\label{fig:hell_umdrehen}
\end{figure}

Es wird klar ersichtlich, dass das so erzeugte Bild nicht scharf erscheint.
Dies kann durch eine Anpassung der Position der Probe korrigiert werden. Ein
Verkleinerung der Distanz zwischen Okular und Probe sorgt dafür, dass wieder
ein scharfes Bild entsteht.

\subsubsection{Kohärenz der Beleuchtung}

Um die Kohärenz der Beleuchtung zu untersuchen wird der Diffusor aus dem
Strahlengang entfernt. Dadurch entsteht ein verschwommenes Bild, welches durch
Anpassung der Distanzen wieder scharf gestellt werden kann. Das so erhaltene
Bild ist in \autoref{fig:hell_koheranz} ersichtlich.

\begin{figure}[H]
	\begin{center}
		% \includegraphics[width =0.5\textwidth]{./figures/advmicroskope/hellmikro-diffusor-scharf.jpg}
	\end{center}
	\caption[Scharfes Bild ohne Diffusor] { Scharfes Bild ohne Diffusor
	}\label{fig:hell_koheranz}
\end{figure}

\subsubsection{Wellenlängenabhängigkeit und chromatische Aberration}

Um die Wellenlängenabhängigkeit der Lichtquelle zu betrachten, wird die
Halogenlampe aus dem Aufbau entfernt und durch farbige LEDs ersetzt. Es ist
darauf zu achten, dass der Diffusor wieder im Strahlengang eingebaut sein muss.
Die Stromversorgung der LEDs wird dabei über den USB Anschluss und den
entsprechenden Adapter sichergestellt. Zunächst wird die blaue LED in den
Aufbau eingebaut und das Bild scharf gestellt. Nun wird die blaue LED durch
eine Rote ersetzt und das entstehende Bild durch Verschieben der Probe erneut
scharf gestellt. Die entsprechende Distanz der Vergrößerung des Abstandes wird
gemessen, was folgenden Wert liefert:

\begin{equation*}
	\SI{6.09(2)}{\mm}
\end{equation*}

\subsubsection{Gesamtvergrößerung}

Um die Gesamtvergrößerung berechnen zu können, muss zunächst die
Sockel-Sockel-Distanz und Länge der Linsen Halterung bestimmt werden. Daraus
kann die Tubuslänge in \autoref{sec:auswertung} bestimmt werden. Die gemessenen
Werte sind:

\begin{equation*}
	d_\text{SockelSockel}=\SI{64.03(2)}{\mm}
\end{equation*}

\begin{equation*}
	d_\text{LensTube}=\SI{38.90(5)}{\mm}
\end{equation*}

\subsubsection{Auflösungsvermögen}

Um das Auslösevermögen zu bestimmen wird der Bereich der Probe betrachtet, an
dem sich die Balken mit den verschiedenen Gruppen und Nummerierungen befinden.

Nun wird jenes Balkenpaar gesucht welches gerade noch als getrennt wahrgenommen
werden kann. Dies ist in \autoref{fig:auflosung} ersichtlich.

\begin{figure}[H]
	\begin{center}
		% \includegraphics[width =0.5\textwidth]{./figures/advmicroskope/hellmikro-aufloesung.jpg}
	\end{center}
	\caption[Abgebildete Balkenpaare, um festzustellen, welche noch als getrennt
		wahrgenommen werden können] { Abgebildete Balkenpaare, um festzustellen, welche
		noch als getrennt wahrgenommen werden können
	}\label{fig:auflosung}
\end{figure}

Im konkreten Fall handelt es sich bei diesen Balken um:

\begin{equation*}
	\text{Balkenpaar} - 91
\end{equation*}

\subsection{Aufbau eines einfachen Dunkelfeld-Mikroskops}

Zunächst wird das Dunkelfeld-Mikroskop, wie bereits in
\autoref{sec:versuchsanordnung} beschrieben, aufgebaut. Beim Aufbau ist
besonders wichtig, darauf zu achten, dass sich alle Komponenten auf einer
Optischen Achse befinden. Wie bereits erwähnt, wird der Abstand zwischen der
achromatischen Linse und der Kamera mit nicht verwendeten Sockeln konstant
gehalten und das gesamte Paket aus Linse und Kamera so lange verschoben, bis
das gesuchte Dunkelfeld-Bild entsteht. Dabei ist zunächst das Hellfeld scharf
und dann durch Vergrößern der Distanz zwischen Präparat und gesamt Paket kommt
man in das Dunkelfeld. Die entsprechende Distanz wird dadurch bestimmt, dass im
Hellfeld ein scharfes Bild entsteht. Dies stellt sich als die effizienteste
Methode heraus. Durch die verstellbare Blende hinter der Linse kann nun der
ursprüngliche Lichtstrahl ausgeblendet werden, wodurch nur noch das Streulicht
aufgefangen wird. Ein so erzeugtes Bild ist in \autoref{fig:dunkelfeld}
sichtbar.

\begin{figure}[H]
	\begin{center}
		% \includegraphics[width =0.5\textwidth]{./figures/advmicroskope/dunkelfeld-grosser_abstand.jpg}
	\end{center}
	\caption[Erzeugtes Bild mithilfe des einfachen Dunkelfeld-Mikroskops] { Erzeugtes Bild
		mithilfe des einfachen Dunkelfeld-Mikroskops
	}\label{fig:dunkelfeld}
\end{figure}

Hier nochmals die reproduzierbare Methodik:
\begin{enumerate}
	\item Fixiere Abstand zwischen Objektive und Okular und bilde eine verschiebbare
	      Einheit mit den beiden Linsen
	\item Finde scharfes Bild im Hellen, durch Verschieben der Einheit
	\item Vergrößere den Abstand zwischen Präparat und Einheit
	\item Dunkelfeld erscheint und dann Scharfstellen durch Verschieben der Einheit
\end{enumerate}

Beim Umgang mit der Computersoftware ist dabei besonders darauf zu achten, ob
die Kameraübertragung problemlos stattfindet, da sich teilweise Probleme in
Form eines plötzlichen Übertragungs-Stopps ergeben haben. Auch ist darauf zu
achten, den Durchmesser der Blende nicht zu ruckartig zu ändern, da sich dabei
teilweise Probleme bezüglich der Belichtung ergeben haben, was die Software
kurzzeitigt ``überfordert''.

\subsubsection{Dunkelfeld-Aufbau in einer Auflicht-Variante}

Um den Aufbau des Dunkelfeld-Mikroskops in einer Auflicht-Variante
durchzuführen, wurde sich folgendes Konzept, sichtbar in
\autoref{fig:dunkelfeld_auflicht} ausgedacht. Bei der Auflicht-Variante ist
wichtig, dass die betrachtete Probe parallel zur betrachteten Probe steht. Die
Nummern entsprechen dabei wieder jenen in der Skizze.

\begin{enumerate}
	\item Zunächst wird ein einfallender Lichtstrahl benötigt.
	\item Davor wird eine Blende gegeben um nur äußere Strahlen zu verwenden.
	\item Diese Strahlenreste werden über Spiegel in Richtung Probe gelenkt.
	\item Zur Umlenkung verwendete Spiegel
	\item Die Lichtstrahlen werden mithilfe von Parabolspiegeln auf die Probe gebündelt.
	\item Die betrachtete Probe befindet sich parallel zur Lichtquelle.
	\item Auf der Probe finden Streuungen des Lichts statt.
	\item Diese Streuungen werden mithilfe der Objektivlinse parallel gerichtet.
	\item Der restliche Aufbau entspricht dem bekannten Strahlengang eines herkömmlichen
	      Mikroskops.
\end{enumerate}

\begin{figure}[H]
	\begin{center}
		% \includegraphics[width =\textwidth]{./figures/skizze_dunkelfeld_auflicht.jpg}
	\end{center}
	\caption[Möglicher Dunkelfeld-Aufbau in einer Auflicht-Variante] { Möglicher
		Dunkelfeld-Aufbau in einer Auflicht-Variante \\
		1 \dots eingehender Lichtstrahl              \\
		2 \dots Blende                               \\
		3 \dots umgelekte Strahlen                   \\
		4 \dots Spiegel                              \\
		5 \dots Parabolspiegel                       \\
		6 \dots Probe                                \\
		7 \dots gestreutes Licht                     \\
		8 \dots Objektiv-Linse                       \\
		9 \dots ausgehender Lichtstrahl
	}\label{fig:dunkelfeld_auflicht}
\end{figure}

\section{Auswertung}\label{sec:auswertung}

Um zu sehen wie sich die Unsicherheit der Messungen bis in die Ergebnisse
fortpflanzt, ist erweiterte Gauss-Methode verwendet worden. Die Grundlagen
dieser Methode stammen von den Powerpointfolien von
% GUM~\cite{wolfgang_kessel_isobipm-gum_2004}. Für die Auswertung ist die
Progammiersprache Python im speziellen die Pakete \verb#labtool-ex2#,
\verb#pandas#, \verb#sympy#, \verb#lmfit# zur Hilfe genommen worden.
\verb#lmfit# wurde für das Fitten verwendet, \verb#sympy# wurde für symbolische
Manipulation verwendet und die restlichen Pakete für leichteres Handhaben der
Daten. Dies wurde aber alles durch \verb#labtool-ex2# abstrahiert.

Um höchstmögliche Genauigkeit zu garantieren wird erst bei der Darstellung der
Wert in Tabellen gerundet.

\subsection{Abbildung durch eine Sammellinse}

\subsubsection{Lupeneffekt}

Die Brennweite der Linse errechnet sich aus dem gemessenen Abstand und der
halben Dicke der Linse. Daraus ergibt sich eine Brennweite von:

\begin{equation}
	f = b'+ \frac{d_\text{Lupe}}{2} = \SI{31(4)}{\mm}
\end{equation}

\subsubsection{Abbildungsgleichung}

Um die Brennweite anhand der Abbildungsgleichung zu bestimmen, werden die
entsprechenden Werte auf \autoref{eq:linsengleichung} angewendet. Daraus ergibt
sich aus den die zwei Brennweiten, welche nun gemittelt werden:

\begin{equation}
	f = \frac{f_1+f_2}{2} = \SI{29.3(4)}{\mm}
\end{equation}

\subsubsection{Bessel-Verfahren}

Um die Brennweite anhand dse Bessel-Verfahrens zu bestimmen, werden die
entsprechenden Werte auf \autoref{eq:bessel} angewendet. Dazu musste jedoch $w$
durch die Unterschiede der Bild- und Gegenstandsweite bestimmt werden indem die
Unterschiede betragsmäßig gemittelt wurden.

\begin{equation}
	w = \frac{|\Delta b|+|\Delta g|}{2} = \SI{29.0(1.1)}{\mm}
\end{equation}

Daraus ergibt sich, mittels \autoref{eq:bessel}, für die Brennweite:

\begin{equation}
	f =  \SI{29.30(0.13)}{\mm}
\end{equation}

\subsection{Das 1-Linsen-Mikroskop nach van Leeuwenhoek}
Unter Verwendung von \autoref{eq:effektive_fokuslange} kann mittels dem
Brechungsindex $n=\num{1.518}$ und dem Durchmesser der Kugellinse $d$ die
effektive Brennweite berechnet werden. Desweitern lässt sich dann durch
\autoref{eq:LeeuwenhoekVergroeserung} die Vergrößerung $M$
% bestimmen.~\cite{banzer_advanced_2022}%todo \cite{}

\begin{enumerate}
	\item Brennweite $f=\SI{1.832(0.004)}{\mm}$ und Vergrößerung $M=\SI{136.5(3)}{}$ bei
	      einem Kugellinsendurchmesser von \SI{2.5}{\mm}
	\item Brennweite $f=\SI{4.652(3)}{\mm}$ und Vergrößerung $M=\SI{53.73(5)}{}$ bei
	      einem Kugellinsendurchmesser von \SI{6.35}{\mm}
\end{enumerate}

\subsection{Hellfeld-Transmissionsmikroskop}

\subsubsection{Gesamtvergrößerung}

Zunächst wird die Distanz vom Sockel zu Sockel mit der Länge der
Linsenhalterung addiert und die Brennweiten davon subtrahiert, damit die
Tubuslänge $t_0$ berechnet werden kann:

\begin{equation}
	t_0 = d_\text{SockelSockel} + d_\text{LensTube} + d_\text{Sockel} - f_\text{Okular} - f_\text{Objektiv}
\end{equation}

Schlussendlich wird mittels der der Tubuslänge $t_0=\SI{33.3(1.4)}{\mm}$ die
Gesamtvergrößerung $M$ des Mikroskops berechnen.

\begin{equation}
	M=\frac{-t_0 f_\text{Bezugssehweite}}{f_\text{Okular} f_\text{Objektiv}} = \num{-4.8(0.4)}
\end{equation}

\subsubsection{Auflösungsvermögen}

Das Auflösungsvermögen kann aus der entsprechenden Gruppen und Elementnummer
anhand \autoref{eq:auflosungsvermogen} bestimmt werden. Dies ergibt schließlich
folgenden Wert:

\begin{equation}
	\SI{512}{\strich\per\mm}
\end{equation}

Durch Bildung des Kehrwert für ein Balkenpaar entsteht für das Mikroskop ein
Auflösungsvermögen von $\SI{1.953}{\um}$.

\subsection{Aufbau eines einfachen Dunkelfeld-Mikroskops}

Bezüglich der Dunkelfeld-Mikroskopie ist keine Auswertung notwendig. Ein
Vergleich der erzeugten Bilder durch das Hellfeld-Transmissionsmikroskop mit
dem einfachen Dunkelfeld-Mikroskop zeigt, dass die erhaltenen Bilder durch das
einfache Dunkelfeld- von der Belichtung her genau umgekehrt sind.%(\autoref)%(\autoref{label})

\section{Diskussion}\label{sec:diskussion}

\subsection{Abbildung durch eine Sammellinse}

Alle drei Varianten liefern, wenn alle Distanzen berücksichtigt werden, eine
Brennweite von in etwa \SI{29}{\mm}. Was ein Indiz dafür ist, dass die
tatsächliche Brennweite wirklich diesem Wert entspricht.

\subsection{Das 1-Linsen-Mikroskop nach van Leeuwenhoek}

%Ab welchem Brechungsindex liegt der Fokuspunkt nicht mehr außerhalb der Kugellinse?

Der Vorteil des 1-Linsen-Mikroskop nach van Leeuwenhoek liegt darin, dass nur 1
Linse verwendet werden muss. Aufgrund des Aufbaus wird der Strahlengang auch
nicht durch die Appertur sondern durch das Auge begrenzt. Ein weiterer Vorteil
ist, dass sich der Herstellung einer Kugellinse, vor allem in historischer
Sicht, als einfacher erweist als die einer herkömmlichen Sammellinse. Es ist
auch sehr beeindruckend, dass sich durch diesen recht simplen Aufbau bereits
eine Vergrößerung von über 100 realisieren lässt. Jedoch ist die Handhabung
eines normalen Mikroskops deutlich bequemer. Auch kann die Vergrößerung durch
Anpassen der entsprechenden Distanzen zwischen den Linsen oder Verwendung
anderer Brennweiten nochmals deutlich vergrößert werden. Auch ist der
Abbildungsfehler, hervorgerufen durch die sphärische Abberation, bei der
Kugellinse deutlich größer, was auch ein Betrachten von
\autoref{fig:van_leeuwenhoek_bild} zeigt.

\subsection{Hellfeld-Transmissionsmikroskop}

Aufgrund des Schlechten Aufbaus ist nur eine Vergrößerung von circa \num{5}
geschafft worden, und das Auflösungsvermögen des Mikroskop ist im Vergleich zu
Objektiven anderer Mikroskope schlecht. Rechnet man das Auflösungsvermögen
\SI{1.953}{\um} auf eine Numerische Aperture um erhält man eine $N.A. =
	\SI{0.14}{}$.

Ein Verbesserungsvorschlag hierzu wäre, eine bessere Anpassung der verwendeten
Linsen. So könnte durch die Verwendung von anderen Linsen, deren Brennpunkte
besser auf die entsprechenden Distanzen abgestimmt sind, eine deutlich bessere
Vergrößerung erreicht werden.

\subsection{Aufbau eines einfachen Dunkelfeld-Mikroskops}

Beim Betrachten des erzeugten Dunkelfeld-Bildes in \autoref{fig:dunkelfeld}
wird deutlich sichtbar, dass vor Allem leichte Verunreinigungen gut sichtbar
gemacht werden, was mit der Theorie der Dunkelfeldmikroskopie übereinstimmt. Es
wird auch sichtbar, dass ein Tausch bezüglich der hellen und dunklen Bereiche
im Vergleich zur Hellfeldmikroskipie stattgefunden hat. Dies deckt sich
ebenfalls mit den Erwartungen. Es ist auch zu erwähnen, dass Präzision
gefordert ist, um ein Dunkelfeld-Bild zu erzeugen, damit wirklich alle Bereiche
des Lichts ausgeblendet werden, das Streulicht aber durch die Kamera scharf
abgebildet wird.

\section{Zusammenfassung}\label{sec:zusammenfassung}

Hier werden nochmals alle Ergebnisse dieser Experimentenfolge aufgelistet.
%Wobei die meisten zu erstellenden Diagramme Aufgrund der Länge der
%\autoref{sec:auswertung} entnommen werden sollen.

\subsection{Abbildung durch eine Sammellinse}

Hier ist die, in diesem Experiment durch verschiedene Methoden ermittelte,
Brennweite der, zu untersuchenden, Sammellinse nochmals angeführt:

\begin{enumerate}
	\item Lupeneffekt: $f = \SI{31(4)}{\mm} $
	\item Abbildungsgleichung: $f = \SI{29.3(4)}{\mm} $
	\item Bessel-Verfahren: $f = \SI{29.30(0.13)}{\mm} $
\end{enumerate}

\subsection{Das 1-Linsen-Mikroskop nach van Leeuwenhoek}

Hier sind die, in diesem Experiment ermittelte, Brennweite und Vergrößerung
dem, von uns gebauten, van Leeuwenhoek-Mikroskops nochmals angeführt:

\begin{enumerate}
	\item Brennweite $f=\SI{1.832(0.004)}{\mm}$ und Vergrößerung $M=\SI{136.5(3)}{}$ bei
	      einem Kugellinsendurchmesser von \SI{2.5}{\mm}
	\item Brennweite $f=\SI{4.652(3)}{\mm}$ und Vergrößerung $M=\SI{53.73(5)}{}$ bei
	      einem Kugellinsendurchmesser von \SI{6.35}{\mm}
\end{enumerate}

\subsection{Hellfeld-Transmissionsmikroskop}

Bei der Drehung der Linse wird klar ersichtlich, dass sich dadurch die
Brennweite verändert und daher die Distanzen neu angepasste werden müssen, um
ein scharfes Bild zu erzeugen.

Auch wird durch das entfernen des Diffusors gut ersichtlich, dass sich dieser
ebenfalls auf den entsprechenden Strahlengang auswirkt.

Durch die Verwendung der Verschiedenfarbigen LEDs wird der Abbildungsfehler der
chromatischen Abberation klar ersichtlich. Dies bedeutet, dass der Brennpunkt
von der Wellenlänge des betrachteten Lichtstrahls abhängig ist. Im konkreten
Fall bedeutet dies, dass das kurzwelligere, blaue Licht einer anderen
Brennweite entspricht, als das langwelligere Rote.

Die berechneten Werte für die Vergrößerung und das Auflösungsvermögen sind im
Folgenden aufgelistet.

\begin{enumerate}
	\item Gesamtvergrößerung des Mikroskops: $M = \num{-4.8(0.4)}$
	\item Scharfe Balkenpaare per Millimeter: \SI{512}{\strich\per\mm}
	\item Auflösungsvermögen des Mikroskops bei weißem Licht: $d = \num{1.953}$
\end{enumerate}

\subsection{Aufbau eines einfachen Dunkelfeld-Mikroskops}

Die Verwendung des Dunkelfeld-Mikroskops zeigt eine Invertierung der
Helligkeiten. Dies bedeutet, dass die beim Hellfeldmikroskops
``Licht-blockierenden'' Stoffe, die dunkel erscheinen, nun im Dunkelfeld Hell
wirken. Dies lässt sich durch die Streueigenschaften dieser Stoffe erklären.

\newpage
\printbibliography
%todo literatur
\listoffigures
\listoftables
\end{document}
